\documentclass[a4paper,11pt]{article}
%\documentclass[a4paper,10pt]{scrartcl}

\usepackage[utf8]{inputenc}
\usepackage{listings}
\usepackage{color}
\usepackage{geometry}
\usepackage{url}
\usepackage{graphicx}
\usepackage{parskip}

\geometry{a4paper, portrait, margin=1.2in}

\definecolor{mygreen}{rgb}{0,0.4,0}
\definecolor{mygray}{rgb}{0.5,0.5,0.5}
\definecolor{mymauve}{rgb}{0.58,0,0.82}

\newcommand*\justify{
  \fontdimen2\font=0.4em % interword space
  \fontdimen3\font=0.2em % interword stretch
  \fontdimen4\font=0.1em % interword shrink
  \fontdimen7\font=0.1em % extra space
  \hyphenchar\font=`\-   % allowing hyphenation
}

\title{PyFabil}
\author{Alessio Magro}
\date{}

\pdfinfo{%
  /Title    (Python FpgA Board Interface Layer)
  /Author   (Alessio Magro)
  /Creator  ()
  /Producer ()
  /Subject  ()
  /Keywords ()
}

\begin{document}
\sloppy
\maketitle

This document servers as a rough user-guide for PyFabil and related software 
tools. These include a C-like API for communicating with digital boards, a 
Python wrapper with extended functionality for interactive debugging and 
scripting, as well as a C++ server for communication with higher software 
layers. The latter program is primarily designed to be used by high level 
client applications and is based on a Protobuf over ZeroMQ interface. This 
document is not meant to provide a detailed technical discussion on the 
software's implementation. The tools are still in active development, and many 
feature are not implemented yet. This document will be continually updated as 
new features are introduced. 

\section{Introduction}

The primary aim of pyfabil is to provide a single access points to multiple 
digital boards. In this way, the same function signatures can be used to access 
any board type and instance. All communication protocol as well as 
device and register handling details are hidden to the client. The end user 
does not need to know whether the underlying communcation is performed using a 
UDP or TCP-based protocol, or whether the list of registers is loaded through 
an XML file or from a the programmed firmware itself. Most of these are 
implemented in a C++, primarlity for speed. The higher level Pyhon interface 
provides an abstracted view of the board. 

Three boards are currently supported:
\begin{itemize}
 \item UniBoard
 \item ROACH and ROACH2
 \item iTPM (Italian Tile Processing Module, once of the candidates for the SKA 
LFAA digital backend)
\end{itemize}

\begin{figure}[t]
\centering
\includegraphics[scale=0.8]{pyfabil.png}
\caption{PyFabil overall design}
\label{pyfabil_design}
\end{figure}

Figure \ref{pyfabil_design} depicts the overall design of the Python layer. TThe 
interface block encapsulates the API provided by the FPGA Board Access Layer 
shared library, allowing higher level components to call C functions directly. 
This can be used on its own, allowing software developers to implement their own 
abstract functionality. The \texttt{Defs} block is a placeholder for 
package-wide declarations, including enumeration, decorators, Exceptions, and so 
on. The \texttt{FPGABoard} is an abstract superclass which implements common 
functionality, available to all board types. These include :
\begin{itemize}
 \item Connecting to and disconnecting from a board
 \item Load and unload plugins
 \item Calling the wrapped C++ functions with default parameters
 \item Keeping the internal state of the instance (for example, FPGA(s) 
programmed, connected, faulty)
 \item Provide helper functions, such as to search for a particular register or 
device name
 \item Automatic handling of register bitfields (all masking and shifting is 
performed internally if this information is available)
\item Significant error checking and logging routines
\end{itemize}

For each new board type an new class should be implemented which inherits from 
\texttt{FPGABoard}. Any method defined in the superclass can be re-implemented 
(overridden), however one should ensure that any book-keeping defined in the 
parent's class are performed here as well. Ideally, the parent's method is 
called directly after board-specific functionality is implemented. This 
generally takes the form of: \texttt{super(ClassName, self).method\_name()}. 

When firmware is loaded onto FPGAs the functionality of the board changes. 
Firmware-specific initialisation, checks, monitoring and control have to be 
performed. Generally, this is accomplished by writing custom scripts which 
utilize the underlying protocol and board wrappers. These scripts are then 
usually directly integrated into the set of scripts used during the 
operational lifetime of the instrument. These scripts are disjoint, can be 
written by different people, and have no consistency, unless this is 
explicitly enforced. The latter constraint is of particular concern when 
designing a general monitoring and control framework with multiple states, 
modes and status check routines. 

To address this issue, pyfabil includes a firmware block \texttt{Plugin} 
mechanism. When a developer writes a new firmware block (where this block can in 
effect be the entire design), a plugin associated with this firmware must be 
implemented. This plugin must subclass the FirmwareBlock abstract class, define 
a number of parameters (implemented as decorators), implement the abstract 
methods defined in the superclass as well as any custom functionality. These 
plugins can then be dynamically loaded to any board instance during runtime. The 
abstract methods make sure that all plugins provide sufficient functionality to 
be able to perform basic tests, initialise the firmware, routinely check the 
status of the running firmware and perform tests and diagnostics (assuming that 
these were properly implemented). The custom functionality implemented in these 
plugins can also be directly called. The list of available plugins is 
dynamically checked by the python wrapper, and when loading a plugin, it checks 
whether it is compatible with the board to which it is being loaded. It is up to 
the plugin developer to make sure that the loaded firmware on the board is 
correct (for example, by checking firmware name, version, and so on). It is 
generally desirable to be able to configure these plugins automatically. For 
example, during initialisation, a number of registers need to be set and checked 
for a particular observation. This can either be performed explicitly by the 
user, or by providing a configuration file which performs this automatically. 
Plugins will be covered in greater detail in section XXX.

The Python wrapper, together with the FPGA Board Access Layer, can either be 
used as a standalone tool, or in combination with other tools in more complex 
systems. In both cases, a standard and automatic way of configuring all the 
boards and firmware forming part of a deployment is desirable. A deployment of 
multiple boards and firmware will be referred to as an instrument. An instrument 
is composed of multiple boards, which can be of different types, each of which 
can have a number of firmware loaded (depending on the number of FPGAs on the 
board). Each board can have a different role in the setup. Instead of having to 
manually configure all boards and firmware, an \texttt{Instrument} management 
class is provided which, given a configuration file, will:
\begin{itemize}
 \item Parse the configuration file 
 \item Create all board instance, which can be of different types
 \item Connect to these boards
 \item Load the respective firmware on each board
 \item Load required plugins on each board
 \item Call the initialization routine on each board
 \item Call the initialization routine for each plugin
 \item Exposes functions for checking the status of the entire instrument, as 
well a drill-down view of the status of each board and plugin
\end{itemize}

The configuration file takes the form of an XML file. Since this feature is 
still in development, additional information will be provided in a later 
version of this document.

\section{Generic FPGA Board}

The main python package is called \texttt{pyfabil}. It contains a number of 
sub-packages:
\begin{description}
 \item[base] Implementations of base functionality, including the wrapper to 
the C++ shared library, the definitions file and a set of utilities
 \item[boards] Implementations of FPGABoard class as well as all digital board 
sub classes
 \item[instruments] Contains the generic instrument script and a placeholder 
for any custom instrument scripts 
 \item[plugins] Contains a sub-package for each board type, each having a set 
of plugins implemented for the particular board
 \item[tests] Location of any unit tests and test scripts
\end{description}

The \texttt{FPGABoard} implements the basic functionality for interfacing with 
a digital board. The following list defines the functions available in this 
class, together with a brief descriptions of how they would be used. Since an 
instance of the this class cannot be created, an instance of the \texttt{TPM} 
will be used in the examples below. To following snippet can be used to 
instantiate a TPM instance:


\lstset{ %
  backgroundcolor=\color{white},   
  basicstyle=\footnotesize,        
  breakatwhitespace=false,         
  breaklines=true,                 
  captionpos=b,                    
  commentstyle=\color{mygreen},    
  deletekeywords={...},            
  escapeinside={\%*}{*)},          
  extendedchars=true,              
  frame=none,                    
  keepspaces=true,                 
  keywordstyle=\color{blue},       
  language=Python,                       
  numbers=none,                    
  numbersep=5pt,                   
  numberstyle=\tiny\color{mymauve}, 
  rulecolor=\color{black},         
  showspaces=false,                
  showstringspaces=false,          
  showtabs=false,                  
  stepnumber=2,                    
  stringstyle=\color{mymauve},     
  tabsize=2,
  title=\lstname
}

\begin{lstlisting}[belowskip=-1 \baselineskip]
 from pyfabil import TPM
 
 # Instantiate TPM instance
 tpm = TPM()
 
 # Instantiate TPM instance and connect to board
 tpm = TPM(ip="127.0.0.1", port=10000)
\end{lstlisting}

The first statement imports the \texttt{TPM} class from \texttt{pyfabil} (Note 
that even though the TPM class is defined in \texttt{pyfabil.boards.tpm}, 
helper imports are defined in the \_\_init\_\_ file of the root package). The 
next statement then create a new instance of class \texttt(TPM). When an IP and 
port are specified, as in the third statement, the \texttt{connect} method in 
the library is called, setting up all the required internal data structures. It 
will also check whether it can communicate with the destination IP and port (in 
this case using the UCP protocol). A TPM simulator is also provided in 
\texttt{pyfabil.tests.tpm\_simulator} in cases where a TPM is unavailable. This 
can be run in a separate console, and it will bind itself to all local IP 
addresses and wait for read and write requests. \texttt{FPGABoard} will keep 
an internal status for the device and all FPGAs defined for the board. If no 
error occurs, the status for a connected board will be \texttt{Status.OK}, 
otherwise if a network error occurs \texttt{Status.NetworkError}. The 
constructor will also set up logging and get a list of all available plugins 
which can be loaded for the board instance. 

The status of the digital board upon connection depends on the board type as 
well as whether it (FPGAs on the board) has been programmed or not. Generally, 
if a firmware is already programmed the instance will get the query the 
register list and populate internal structures. These checks and the population 
of the register list is board-specific, so this functionality is implemented in 
the board classes. Generally, when connecting to a board which has already been 
programmed, the register is automatically generated and the user can then 
interact with them.

The programmed firmware can be changed by using the \texttt{load\_firmware} 
method, as per the following snippet. This will re-program the specified device 
and re-load the register list. Note that the behavior of the command is 
board-specific, so please consult the appropriate sections in this document.

\begin{lstlisting}[belowskip=-1 \baselineskip]
from pyfabil import Device
tpm.load_firmware(Device.FPGA_1, firmware="custom_design")
\end{lstlisting}

The first argument defines on which FPGA on the board the firmware will be 
loaded. The TPM has two FPGAs, and this statement will load the firmware 
labelled ``custom\_design'' on FPGA 1 (or ID 0). This call will populate 
all internal data structures containing register and memory block mapping 
information, such that read and write operations can immediately be issued by 
the user. An additional call, \texttt{get\_firmware\_list} can be used to list 
the firmware available on the board (generally on Flash or over an NFS-mounted 
directory). Additional functions are also provided, defined in section XXX.

Once the board is connected and the register list is loaded, operations on 
registers, devices and memory block can be performed. Three types of operations 
can be performed:
\begin{itemize}
 \item Read from and write to a memory address
 \item Read from and write to a register
 \item Read from and write to a memory address on a device (such as an SPI 
device)
\end{itemize}
A read and write method for each operation type is provided. Indexed access to 
the board instance can also be used to indirectly call these methods. Use 
indexed access results in less clutter in the code and makes it more readable. 
However, indexed access cannot be used in all cases, for example if a number of 
words or an offset needs to be provided, then the methods have to be called 
directly. The following snippet demonstrates these for a register (or memory 
block). The two sets of statements here are equivalent.

\begin{lstlisting}[belowskip=-1 \baselineskip]
tpm.write_register('fpga1.register_name', 0x25)
val = tpm.read_register('fpga1.register_name', n = 1)

tpm['fpga1.register_name'] = 0x25
val = tpm['fpga1.register_name']
\end{lstlisting}

The first argument is the register name on which the operation will be 
performed. The structure of this name is board-specific. Generally, the device 
name with which the register is associated will be pre-pended to the actual 
name, such that identical register names on different FPGAs can be identified. 
The list of register can be grouped into components (as is the case for the 
TPM). The full register name must match the internal representation of the 
register (as stored in the register dictionary). The list of register names can 
be acquired by calling \texttt{list\_register\_names}. Additional information 
for each register can also be analyzed by calling \texttt{get\_register\_list}. 
A user-friendly print out of the register list can be displayed by the 
statement:
\begin{lstlisting}[belowskip=-1 \baselineskip]
print tpm
\end{lstlisting}
which will generated an output similar to:
\begin{lstlisting}
 Device  Register                           Address        Bitmask
------  --------                           -------        -------
FPGA 1	regfile.block2048b                 0x1000L        0xFFFFFFFF
FPGA 1	regfile.date_code                  0x0L           0xFFFFFFFF
...
Board   regfile.ada_ctrl.ada_a_ada4961     0x30000010L    0x00000008
Board   regfile.ada_ctrl.ada_fa_ada4961    0x30000010L    0x00000008
... (redacted)
\end{lstlisting}

\begin{lstlisting}
 # Read register values from specified register
 # device   - Device to which this register belongs to
 # register - Register name/mnemonic
 # n        - Number of words to read
 # offset   - Address offset to start reading from
 # Returns: Value or list of values
 def readRegister(device, register, n = 1, offset = 0)
 
 # Write values to specified register
 # device   - Device to which this register belongs to
 # register - Register name/mnemonic
 # values   - Value or list of values to write
 # offset   - Address offset to start reading from
 # Return:  Success or Failure
 def writeRegister(device, register, values, offset = 0)
 
 # Read values from specified memory address
 # address  - Memory address
 # n        - Number of words to read
 # Returns: Value or list of values
 def readAddress(address, n = 1)
 
 # Read register values from specified register
 # address  - Memory address
 # values   - Value or list of values to write
 # Returns: Success or Failure
 def writeAddress(address, values):
\end{lstlisting}

The first two functions defined above require both the \texttt{device} and 
\texttt{register} name to be able to compute the memory address. In all cases, 
checks are performed by the wrapper to make sure that requests don't go out of 
bounds (for example, the length of the list of values to write are larger than 
the size of the register, the register has write permissions in case of writes, 
and so on). The list of registers which have been loaded can be displayed 
either 
with the \texttt{listRegisterNames()} function call, or with the following 
statement: \texttt{print tpm}, where tpm is a \texttt{TPM} class instance. This 
will print out the list of register with additional information, as shown below:

\begin{lstlisting}
 Device  Register                           Address        Bitmask
------  --------                           -------        -------
FPGA 1	regfile.block2048b                 0x1000L        0xFFFFFFFF
FPGA 1	regfile.date_code                  0x0L           0xFFFFFFFF
FPGA 1	regfile.debug                      0x10L          0xFFFFFFFF
FPGA 1	regfile.jesd_channel_disable       0xcL           0x0000FFFF
FPGA 1	regfile.jesd_ctrl.bit_per_sample   0x4L           0x000000F0
FPGA 1	regfile.jesd_ctrl.debug_en         0x4L           0x00000001
FPGA 1	regfile.jesd_ctrl.ext_trig_en      0x4L           0x00000002
FPGA 1	regfile.reset.global_rst           0x8L           0x00000002
FPGA 1	regfile.reset.jesd_master_rstn     0x8L           0x00000001
Board   regfile.ada_ctrl.ada_a_ada4961     0x30000010L    0x00000008
Board   regfile.ada_ctrl.ada_fa_ada4961    0x30000010L    0x00000008
Board   regfile.ada_ctrl.ada_latch_ada4961 0x30000010L    0x00000008
... (redacted)
\end{lstlisting}

Indexed access provides a more Pythonesque methods of accessing registers. The 
three statements below will perform the same operation. All of them will read 
all the values for memory block \texttt{regfile.block2048b}, which is defined 
for \texttt{FPGA1}. The first statement specifies the full register name as 
provided by the access layer. In cases where the same full name is specified 
for both FPGAs (same firmware is loaded) the device name can also be specified 
as in the second example. In the third statement, the memory address is 
provided immediately, such that the \texttt{readAddress} function is called 
instead of \texttt{readRegister}.

\begin{lstlisting} 
 1. tpm['regfile.block2048b']
 
 2. tpm['fpga1.regfile.block2048b']
 
 3. tpm[0x1000]
\end{lstlisting}

Similar statement can also be defined for writing values. In the examples 
below, the memory region defined for regfile.block2048b (512 words) will be 
initialized with \texttt{1} for all values. Note that in order to write to an 
address offset, the function calls must be used directly.

\begin{lstlisting}
 1. tpm['regfile.block2048b'] = [1] * 512
 
 2. tpm['fpga1.regfile.block2048b'] = [1] * 512
 
 3. tpm[0x1000] = [1] * 512
\end{lstlisting}

The wrapper and underlying library also handle bitmasks automatically, as shown 
in the example below, where \texttt{regfile.bitregister.bit8} at address 
\texttt{0x10001000} has a bitmask of \texttt{0x00000008}

\begin{lstlisting}
 1. print tpm['regfile.bitregister.bit8']      Output = 0

 2. tpm['regfile.bitregister.bit8'] = 1
 
 3. print tpm['regfile.bitregister.bit8']      Output = 1
    
 4. print tpm[0x10001000]                      Output = 8
\end{lstlisting}

Reads from and writes to a bitfield will automatically be shifted such that the 
value belonging to the bitfield itself are displayed. Writes to a bitfield are 
performed as a read-modify-write operation in the access layer library. The 
Python wrapper also provides some additional features, as show below:

\begin{lstlisting}
 
            # Use regular experssions to search for a particular register 
            # from list of register, returning a dictionary for each 
            # match. Can also print out the results
 Statement: tpm.findRegister("*block2048b*", display = True)
 Output:    regfile.block2048b:
            ------------------
            Address:       0x1000L
            Type:          RegisterType.FirmwareRegister
            Device:        Device.FPGA_1
            Permission:    Permission.ReadWrite
            Bitmask:       0xFFFFFFFF
            Bits:          32
            Size:          512
            Description:   For testing
            
            # Return number of registers
 Statement: len(tpm)
 Output:    32
 
\end{lstlisting}

Once all operations have been performed on a TPM, the the script should 
disconnect from the board:

\begin{lstlisting}
 tpm.disconnect()
\end{lstlisting}


\section{iTPM}

\subsection{XML Memory Map}
\label{XML}

The XML file provides a mapping between register and memory block 
names/mnemonics to memory address on the board. In the TPM's case, the XML file 
can contain the mapping for up to three 'devices': the CPLD (or board), the 
firmware loaded on the first FPGA and the firmware loaded on the second FPGA. 
The same firmware can be loaded on both FPGAs. The register in a firmware map 
can be logically split into components (such as channeliser, beamformer, 
calibrator ...), and in turn each register can be composed of multiple 
bit-fields, where disjoint sets of bits have different interpretation in the 
firmware. For this reason, an XML mapping file is composed of a hierarchy of 
four nodes (excluding the root node):
\begin{description}
 \item[Device node] Highest level node, representing the board itself (CPLD) or 
the firmware loaded on the FPGAs. The identifiers for nodes in this level are 
``CPLD'', ``FPGA1'' and ``FPGA2''.
 \item[Component node] Registers for each device can be grouped into 
components, which makes it easier to logically partition the functionality of 
these registers. For example, registers belonging to different firmware 
components can be grouped into different nodes at this level. A base address 
can 
be specified for each component.
 \item[Register node] Each component can be composed of multiple registers or 
memory blocks, each requiring an identifier (the name or mnemonic) and the 
memory-mapped address. Additional attributes can also be specified, such as the 
size in words, access permission, the bitmask and a description. 
 \item[Bit node] A register can be composed of a combination of multiple 
bitfields. These bitfields are specified in the fourth node level, having an 
entry for each bitfield in the register. A bitmask per bitfield must be 
specified.
\end{description}

The XML listing below provides a very concise example of an XML mapping file. 
Each of the node attributes are defined as follows:

\begin{description}
 \item[id] The device, register, memory block or bitfield name/mnemonic. This 
does not need to be unique within the XML file (such that the same firmware can 
be loaded on both FPGAs). The IDs of successive levels in the hierarchy are 
combined to generate the full name in the access layer. For example, the 
full name of the first bit in \texttt{bit\_register}, \texttt{regfile}, 
\texttt{FPGA1} in the example below would be 
\texttt{\justify regfile.bit\_register.bit1} (the registers are partitioned 
internally by device). 

 \item[address] The memory address to which the register is mapped. A base 
address can be specified at all levels (except bitfields). In this case, 
successive base addresses are added together to form the final address. 

 \item[permission] Access permissions for the register or memory block. Allowed 
entries are \texttt{r}, \texttt{w} and \texttt{rw}.

 \item[size] The size in words of the register or memory block. If this value 
is not specified, then the register is assumed to be of size 1 (32-bits). 
Memory blocks can essentially be defined by setting this value. Note that if a 
bitmask is specified for a memory block with size greater than 1, then it is 
applied for all consecutive words.
 
 \item[mask] The bitmask to be applied to the values after reading from or 
before writing to the memory address. 

 \item[description] A textual description for the register.
\end{description}

\lstset{
  language=XML,
  captionpos=b,              
  commentstyle=\color{mygreen},    
  stringstyle=\color{mygreen},
  keywordstyle=\color{blue},
  morekeywords={node,id,address,encoding,mask,permission,size,description}
}

\begin{lstlisting}[caption=XML memory map]
<?xml version="1.0" encoding="ISO-8859-1" ?>

<node>

  <node id="CPLD">
    <node id="regfile" address="0x00000000">
      <node id="register1" address="0x00000010" permission="rw" 
            size="1" mask="0xFFFFFFFF" description="..."/>
      ...
    </node>
    ...
  </node>
  
  <node id="FPGA1" address="0x100000000">
    <node id="regfile" address="0x00001000">
      <node id="bit_register" address="0x00000010" />
        <node id="bit1" mask="0x00000001" permission="rw" 
              description="..." />
        <node id="bit2" mask="0x00000002" permission="rw" 
              description="..." />
        ...
      </node>
      ...
    </node> 
    ...
  </node>
  
  <node id="FPGA2"> ... </node>
  
</node>
\end{lstlisting}

\section{UniBoard}

\section{ROACH}

\pagebreak

\section*{Appendix A - Compilation and Installation}

The Access Layer source code can be retrieved from 
\url{https://github.com/lessju/TPM-Access-Layer.git}. The top directory 
contains four directories: 
\begin{itemize}
 \item \texttt{doc}, which contains some documents, example XML files and this 
user guide
 \item \texttt{python}, which contains the Python wrapper and setup scrip
 \item \texttt{script}, which contains helper scripts
 \item \texttt{src}, which contains the source code for the library and server
\end{itemize}

The \texttt{src} directory in turn contains three directories, one containing 
the source code for the C++ library (\texttt{library}), one containing the 
source code of the server (\texttt{server}) and one containing source code for 
exporting the library as a Windows DLL (experimental, \texttt{windows}). Each 
directory contains a Makefile which can be used to build and install each tool. 

\subsection*{Compiling the library}

Requirements: g++ version 4.0+. Compilation and installation steps:
\begin{enumerate}
 \item \texttt{export ACCESS\_LAYER=/path/to/top/level/access/layer/dir}
 \item \texttt{cd \$ACCESS\_LAYER/src/library}
 \item Edit \texttt{INSTALL\_DIR}, \texttt{LIBRARY\_NAME} and \texttt{GCC} in 
the Makefile so that it can be compiled and installed on your system. Note that 
\texttt{GCC} should point to a C++ compiler
 \item \texttt{make}
 \item \texttt{sudo make install}
\end{enumerate}

\subsection*{Installing the Python Wrapper}

Ideally, the python version should be 2.5 or higher. Older versions can also be 
used, however the \texttt{ctypes} package will need to be installed. 

\begin{enumerate}
 \item \texttt{sudo pip install enum34} (Assuming pip is installed)
 \item \texttt{export ACCESS\_LAYER=/path/to/top/level/access/layer/dir}
 \item \texttt{cd \$ACCESS\_LAYER/python}
 \item \texttt{sudo python setup.py install}
\end{enumerate}


\subsection*{Compiling the server}

The server uses two third party libraries: ZeroMQ and Protobuf (latest versions 
of both). You will need to install these in order to use the server.

\begin{enumerate}
 \item \texttt{export ACCESS\_LAYER=/path/to/top/level/access/layer/dir}
 \item \texttt{cd \$ACCESS\_LAYER/src/server}
 \item \texttt{protoc --cpp\_out=. message.proto}. This only needs to be 
performed once, or when message.proto changes
 \item Edit \texttt{LIBRARY\_DIR} and \texttt{GCC} in the Makefile so that it 
can be compiled on your system.
 \item \texttt{make}
\end{enumerate}

If the access layer library was not installed in a system directory, 
then update \texttt{LD\_LIBRARY\_PATH} to point to the library, otherwise the 
server won't manage to load the library: \texttt{export 
LD\_LIBRARY\_PATH=\$LD\_LIBRARY\_PATH:/path/to/library}.

\pagebreak

\section*{Appendix B - Access Layer API}
\label{API}

\lstset{ %
  backgroundcolor=\color{white},   
  basicstyle=\footnotesize,        
  breakatwhitespace=false,         
  breaklines=true,                 
  captionpos=b,                    
  commentstyle=\color{mygreen},    
  deletekeywords={...},            
  escapeinside={\%*}{*)},          
  extendedchars=true,              
  frame=none,                    
  keepspaces=true,                 
  keywordstyle=\color{blue},       
  language=C,                 
  otherkeywords={*,ID, RETURN, VALUES, UINT, STATUS, 
REGISTER,_INFO,REGISTER,DEVICE,...},         
  numbers=none,                    
  numbersep=5pt,                   
  numberstyle=\tiny\color{mymauve}, 
  rulecolor=\color{black},         
  showspaces=false,                
  showstringspaces=false,          
  showtabs=false,                  
  stepnumber=2,                    
  stringstyle=\color{mymauve},     
  tabsize=2,
  title=\lstname                   
}

\begin{lstlisting}[caption=TPM Access Layer API listing]
// Connect to a board with specified IP and port. Returns unique
// board identifier
ID connectBoard(const char* IP, unsigned short port);

// Disconnect from board with ID id
RETURN disconnectBoard(ID id);

// Reset board (currently not implemented)
RETURN resetBoard(ID id);

// Get board status (currently not implemented)
STATUS getStatus(ID id);

// Get a list of board and firmware registers. A memory map needs 
// to be loaded.
REGISTER_INFO* getRegisterList(ID id, UINT *num_registers);

// Read register value from specified device/register. Number of 
// values and address offset can be specified.
VALUES readRegister(ID id, DEVICE device, REGISTER reg, UINT n, 
                    UINT offset = 0);

// Write register value to specified device/register. Number of 
// values and address offset can be specified.
RETURN writeRegister(ID id, DEVICE device, REGISTER reg, UINT n, 
                     UINT *values, UINT offset = 0);

// Read a number of values from address. To be used for debug mode. 
VALUES readAddress(ID id, UINT address, UINT n);

// Write a number of values to address. To be used for debug mode. 
RETURN writeAddress(ID id, UINT address, UINT n, UINT *values);

// Load (non-blocking) firwmare to device. Behaviour currently unspecified
RETURN loadFirmware(ID id, DEVICE device, const char* bitstream);

// Load (blocking) firmware to device. Behaviour currently unspecified.
RETURN loadFirmwareBlocking(ID id, DEVICE device, const char* bitstream);
\end{lstlisting}

\texttt{ID}, %\texttt{RETURN}, \texttt{STATYS}, \texttt{REGISTER_INFO}, 
\texttt{VALUES}, \texttt{UINT} and \texttt{DEVICE} are defined in {\it 
Definitions.hpp}. This header file also contain a macro to enable or disable 
INFO output during execution.

\end{document}
